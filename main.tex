\documentclass{article}
\usepackage[english]{babel}
\usepackage[a4,top=2cm,bottom=2cm,left=3cm,right=3cm,marginparwidth=1.75cm]{geometry}
\usepackage{amsmath}
\usepackage{graphicx}
\usepackage[colorlinks=true, allcolors=blue]{hyperref}
\title{Your Paper}
\author{You}
\begin{document}
\maketitle
\begin{abstract}
A simple abstract
\end{abstract}

%% a# visualization.R
library(ggplot2)

# Exemple de données
data <- data.frame(
  category = c('A', 'B', 'C', 'D'),
  value = c(3, 12, 5, 18)
)

# Créer une visualisation non triviale
p <- ggplot(data, aes(x = category, y = value)) +
  geom_bar(stat = "identity", fill = "skyblue") +
  theme_minimal() +
  ggtitle("Exemple de visualisation")

# Sauvegarder le graphique
ggsave("visualization.png", plot = p)
dd here some latex code to load the content of  introduction.tex (1 line)	

\bibliographystyle{alpha}
\bibliography{sample}
\end{document}
